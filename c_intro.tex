\chapter{Introducción}
\label{ch:introduccion}

Este trabajo surge de la conferencia “Desafío total: cómo resolver
retos extremos de programación” por D. Rafa Bermúdez y D. Jorge
Cubero, que forma parte del ciclo de conferencias de la semana de la
informática y tuvo lugar el día 24 de Febrero. En ella se expusieron
diferentes problemas de programación y alguna solución para los
mismos. La conferencia aunque fue interesante en cuanto a las
soluciones propuestas para los problemas, me resultó incompleta por no
hablar de técnicas para resolver problemas ni animar a la
participación en concursos y/o retos de programación.

En este trabajo pretendo completar esos ejemplos con diferente
información sobre programación para concursos, tipos de problemas,
técnicas comunes, organización de equipos,etc. Además animar al
posible lector a participar o colaborar con la organización en
concursos de programación. Toda la información sale de la propia
experiencia al haber participado en diferentes concursos online,
internacionales, nacionales y en la propia facultad. En los sucesivos
capítulos describo los diferentes tipos de concurso y algunos ejemplos
de ellos. En mi caso he participado en:

\begin{itemize}
\item SWERC (2012, 2013 y 2014)
\item Contest Tuenti ( 1, 2, 3, 5 y 6)
\item Google Code Jam (2014)
\item Ada-Byron (1 edición)
\item Las 12 uvas (2012 y 2013 antes de ser público. 2014 y 2015 con
  inscripción abierta)
\item Entrenamientos y simulacros locales organizados por Marco
  Antonio Gómez y Pedro Pablo Gómez.
\end{itemize}

Aparte he colaborado con la organización de concursos en la facultad
como son:

\begin{itemize}
\item ProgramaMe (2014, 2015 y 2016)
\item Ada-Byron (2016)
\item Selección de equipos para SWERC (2015)
\end{itemize}

Con esto simplemente quiero reflejar que todo lo desarrollado en este
trabajo es fruto de la experiencia personal en la preparación para
concursos y preparación para entrevistas en empresas tecnológicas que
en los últimos años vienen planteando retos de programación en el
proceso de selección. Y, en el trabajo, también se incluye información
aprendida gracias a los hermanos Steven y Felix Halim, y su trabajo
desarrollado en uno de sus tres libros: “Competitive Programming 3:
The New Lower Bound of Programming Contests”, que se encuentra
referenciado en el último capítulo.
