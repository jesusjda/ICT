\chapter{Conclusiones}
\label{ch:conclusiones}

En este trabajo se han presentado una serie de pautas de cómo
participar en concursos de programación y se han mostrado ejemplos de
algunas de las categorías más comunes. Estos ejemplos eran bastante
sencillos con la intención de que fuera fácil entenderlos y ocupen
poco su solución para no distraer el objetivo final del trabajo, que
es una aproximación al mundo del concurso. No existe una solución o
fórmula mágica para ganar concursos, estos requieren práctica y
estudio, “el concurso se gana en los entrenamientos” decía uno de mis
entrenadores.\\

Además, en la introducción se comenta que muchas empresas empiezan a
plantear retos de similares características, por lo que considero muy
beneficioso para los informaticos estar aficionados a este tipo de
concursos, ya que no solo otorga cierta diversión sino una buena
preparación para el futuro laboral. Sin tener en cuenta lo interesante
que puede ser entender cierto tipo de problemas y la emoción que
genera un concurso.\\

Decidí realizar este trabajo dado que la conferencia se me quedó
corta, no pude percibir en los ponentes la emoción que yo he
encontrado aquí. Por lo que decidí completarla con los tipos de
concursos y con clasificaciones de problemas. Me hubiese gustado
realizar el trabajo más completo aún pero por diversas circunstancias
no he podido lograrlo. Remarcar que toda la información del trabajo
está sacada de la experiencia y guiada por el libro citado en las
referencias. Por eso, puede fallar “el bien documentados” que se exige
para el trabajo.\\

Entre las referencias de este trabajo encontramos el libro ya
mencionado de los hermanos Halim~\cite{halim}, el libro de Thomas
H. Cormen~\cite{cormen} utilizado durante los años de entrenamiento y
estudio, el libro de Miguel A. Revilla~\cite{revilla} que fue el
primer libro de concursos de programación que consulté. Y aparecen
también dos jueces de problemas que nos ofrecen diversos enunciados y
la posibilidad de evaluar nuestras soluciones, el primero, el juez de
la Universidad de Valladolid~\cite{uva} que he utilizado para los
ejemplos de este trabajo, el segundo ``¡Acepta el reto!''~\cite{acr}
ha sido realizado en la Facultad de Informática, bajo el mando de los
hermanos Marco Antonio y Pedro Pablo Gómez, que dada mi afición a los
concursos de programación, me ofrecieron colaborar al comienzo del
proyecto junto con Luis María Costero.
