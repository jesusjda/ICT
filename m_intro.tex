\chapter{Introducción}
\label{ch:intro}

Este trabajo se realiza a partir de la conferencia ``Lean Software
Development'' realizada el día 4 de Abril 2016, por la ponente
Dña. Simona Puica. En ella se expuso la metodología de desarrollo del
software llamada como el título de la conferencia que es utilizada en
la empresa ``itestra''.\\

El trabajo busca ser un complemento a esa conferencia. En él se
presenta una breve historia sobre la evolución del desarrollo del
software y cómo los diferentes avances producen cambios en el
mismo. En el Capítulo~\ref{ch:tradicional} se presentan diferentes
ciclos de vida del software que son utilizados por famosas
metodologías de desarrollo y por las nuevas metodologías conocidas
como ágiles. En el Capítulo~\ref{ch:agil} se presentan dos de estas
metodologías ágiles: Scrum y Extreme Programming que hoy en día se
utilizan junto con la expuesta en la conferencia. Por último, el
Capítulo~\ref{ch:conclusiones} recoge unas breves conclusiones así
como la experiencia personal utilizando diversas metodologías y en
especial el reciente uso de la metodología Scrum en un proyecto
universitario.

\section{Historia de la Ingeniería del Software}
\label{sec:historia}

El software surge, en los años 50, como un añadido a los grandes
computadores. Las ideas iniciales eran de cierta sencillez aunque
requerían de cierta complejidad programarlas por eso se consideraba
que programar era un ``arte'' pero de los de ``andar por casa''. Este
“arte” se realizaba sin ninguna planificación previa, simplemente los
programadores trataban de hacer el programa de la mejor manera posible
realizando cambios hasta obtener el resultado deseado, este resultado
consistía en conseguir expresar algoritmos o soluciones bien conocidas
de una manera eficaz en un lenguaje de programación rudo. Cada
programa lo realiza normalmente una única persona y solía ser ella
misma el usuario final dentro de su organización, por ello los
programas se desarrollan a medida que se iban usando y si surgía un
error era el propio usuario quien lo depuraba y corregía. Por este
mismo motivo, la documentación relacionada a ese software era más bien
escasa.\\

En un segundo periodo, que puede durar hasta la mitad de los años 70,
en el que los computadores evolucionan se introducen conceptos como la
multiprogramación y los sistemas multiusuario lo que lleva a redefinir
la interacción ``hombre-máquina''. Entre otras cosas destaca la
aparición de la interactividad con la máquina en tiempo real, estas
podían recoger, analizar y transformar datos de múltiples fuentes de
manera dinámica. En este periodo ya se trata el software como un
producto y aparecen entidades especializadas en desarrollo del
software. Estas entidades establecen sus propios mecanismos de
desarrollo y planificación de software dando lugar al nacimiento de un
nuevo concepto \emph{``ingenier\'ia del software''} definida en 1972 por
F.L.Bauer como: \emph{``El establecimiento y uso de principios de ingeniería
robustos, orientados a obtener económicamente software que sea fiable
y funcione eficientemente sobre máquinas reales''}. Junto al crecimiento
del desarrollo del software crecía también el tamaño del mismo y su
complejidad, y por tanto aumentaban los errores y la dificultad de
solucionar los mismos por lo que se empezó a establecer como una
actividad necesaria el mantenimiento del software, lo cual resultaba
ser lo que más costes conllevaba.\\

Las computadoras eran capaces de realizar tareas y comunicarse con
otras computadoras compartiendo resultados y reaccionando ante
ellos. Esto nos lleva a una tercera fase en la evolución del software,
en la que comienzan los sistemas distribuidos. Comenzaron a aparecer
las redes tanto de área local como global y se buscaba un acceso
``instantáneo'' a los datos. A finales de este periodo R.Fairley define
la ingeniería del software como: \emph{``La disciplina tecnológica y de
gestión que concierne a la producción y el mantenimiento sistemático
de productos software desarrollados y modificados dentro de unos
plazos estipulados y costes estimados''}. Pero el uso de las
computadoras seguía siendo académico e industrial.\\

La aparición de los microprocesadores produce que la computadora se
extienda a los hogares para uso personal y de ocio. Incluso se
incluyen los microprocesadores en otros elementos como son los
electrodomésticos y se desarrolla software específico para estos. Esta
cuarta fase da lugar a la búsqueda de la estandarización y como
referente, el IEEE define en 1990 la ingeniería del software como:
\emph{``(1) La aplicación de un enfoque sistemático, disciplinado y
  cuantificable del desarrollo, la operación y el mantenimiento del
  software. (2) El estudio de enfoques tales como (1)''}. Esta fase
podría asemejarse al segundo periodo llevando lo industrial y
académico a lo personal y ocio. Por lo que las técnicas de desarrollo
aunque evolucionan siguen teniendo en cuenta los mismos
fundamentos. Algunas de las técnicas utilizadas en el desarrollo de
software en los
periodos dos, tres y cuatro se explican en el Capítulo~\ref{ch:tradicional}.\\

Es en el nuevo milenio cuando, con la globalización de Internet,
encontramos una revolución en el desarrollo del software. Se busca
principalmente sacar productos al mercado cuanto antes, manteniendo un
alto nivel de fiabilidad y calidad. El desarrollo se basa en
prototipos y en avances incrementales. Un software triunfa si es el
primero en realizar una funcionalidad, no hace falta que lo haga
perfecto si gana una masa de usuarios suficiente es difícil que una
aplicación que realice la misma tarea de mejor modo le quite su
mercado. Es por esto que nacen las nuevas metodologías de desarrollo
ágil, que como su nombre indica consiguen obtener resultados
rápidamente.
